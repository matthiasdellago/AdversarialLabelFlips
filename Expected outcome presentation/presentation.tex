% !TEX encoding = UTF-8 Unicode
\documentclass[10pt,usepdftitle=false,aspectratio=169]{beamer}
\usepackage[left]{muibkitsec}
\usepackage{listings}

\usepackage{microtype}
\usepackage{graphbox}
\usepackage{booktabs} 

\usepackage{amsmath,amssymb,amsfonts,amsthm,mathtools}
\usepackage{algorithmic}
\usepackage{textcomp}
\usepackage{xcolor}
\usepackage{diagbox}
\usepackage{float, multirow}
\usepackage{tikz, pgfplots}
\usepackage{tikzsymbols}
\usetikzlibrary{spy,matrix}
\usepackage{subcaption}
\usepgfplotslibrary{groupplots}
\pgfplotsset{compat=newest}

% ------------------------------------------------------------------------
\title{Adversarial Label Flips}
%\subtitle{modified new beamer template}
\author{Matthias Dellago \& Maximilian Samsinger}
\date{22 January 2021}

% ------------------------------------------------------------------------

\begin{document}
\DeclarePairedDelimiter\abs{\lvert}{\rvert}%
\DeclarePairedDelimiter\norm{\lVert}{\rVert}%
\DeclarePairedDelimiter\ceil{\lceil}{\rceil}
\DeclarePairedDelimiter\floor{\lfloor}{\rfloor}

\begin{frame}[plain]
	\maketitle
\end{frame}	

\begin{frame}[fragile]
	\frametitle{Introduction}
		\begin{block}{Evasion Attack}
			As seen in Lecture. Insert lecture slides or explain anew?
		\end{block}
\end{frame}

\begin{frame}[fragile]
	\frametitle{Introduction}
	\begin{block}{Confusion Matrix}
		  \begin{table}
			\setlength{\extrarowheight}{2pt}
			\begin{tabular}{cc|c|c|c|}
				& \multicolumn{1}{c}{} & \multicolumn{2}{c}{Categorised as}\\
				& \multicolumn{1}{c}{} & \multicolumn{1}{c}{Dog}  & \multicolumn{1}{c}{Cat} & \multicolumn{1}{c}{Plane} \\\cline{3-5}
				\multirow{3}*{Adverserial Example}  & Dog & 0.0 & ? & ?\\\cline{3-5}
				& Cat & ? & 0.0 &  ? \\\cline{3-5}
				& Plane & ? & ? &  0.0 \\\cline{3-5}
			\end{tabular}
		\end{table}
	\end{block}
\end{frame}

\begin{frame}[fragile]
	\frametitle{Remove}
	Summarize what \#5 is about and recap what the students need to know. (Maybe needs an extra slide)
	Explain (or visualize?) confusion matrix.
\end{frame}

\begin{frame}[fragile]
	\frametitle{Hypothesis}
			\begin{block}{Uniform Distribution?}
				Is post-attack label uniformly distributed over all other labels (null hypothesis) or not?
			\end{block}
			\begin{block}{Reasons?}	
				\begin{itemize}
					\item If uniform, why? If not, why not? Possible relationships between different classes.
					\item Probably intractable, but interesting.
				\end{itemize}
			\end{block}
\end{frame}

\begin{frame}[fragile]
	\frametitle{Methods}
		\begin{block}{Datasets}
			MNIST, Fashion MNIST, CIFAR-10
		\end{block}
		\begin{alertblock}{Alert block}
			ResNet-18 for CIFAR-10.
			Some simple convolutional neural network for MNIST \& Fashion MNIST.
		\end{alertblock}
		\begin{alertblock}{Attacks}
			FGSM and PGD
		\end{alertblock}

\end{frame}

\begin{frame}[fragile]
	\frametitle{Stretch goals}
	\begin{itemize}
		\item Study natural adversarial examples
		\item Look for applications (attacker and defender)
		\item More attacks and/or architectures
	\end{itemize}
\end{frame}


\begin{frame}[fragile]
	\frametitle{Brainstorming slide (will be removed)}
	\begin{columns}
		\begin{column}{.5\columnwidth}
			\begin{alertblock}{What do you want to achieve till the end of semester?}
				\begin{enumerate}
					\item Investigate relationship between ground truth labels and predicted label of adversarial examples. (Maybe formulate as null hyposisis: No correlation)
					\item Github repo for reproducibility.
					\item Max. Learn PyTorch
					\item Matthias. Learn ML  
				\end{enumerate}
			\end{alertblock}
		\end{column}
		\begin{column}{.5\columnwidth}
			\begin{block}{Why is your topic relevant?}
				Contribution to basic research. 
			\end{block}
		\end{column}
	\end{columns}
\end{frame}



\end{document}
